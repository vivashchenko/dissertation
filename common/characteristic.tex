
{\actuality} 

С развитием компьютерных технологий возможности численного моделирования значительно увеличились,
сделав возможным исследовать с помощью численного эксперимента как сложные мультифизичные задачи,
так и редкие и мелкомасштабные локальные явления.
%
В данной работе метод прямого численного моделирования применяется к исследованию трех различных задач,
относящихся к сдвиговым течениям как переменной, так и постоянной плотности.
%
Ниже приведен краткий обзор современного состояния проблематике по каждой из задач, рассматриваемых в работе.

%--------------------------------------------------------------------------------------------
%
Турбулентные течения встречаются повсеместно в природе и промышленности. 
%
В настоящее время в решении задач гидро- и аэродинамики плотно вошли методы численного моделирования. 
%
Еще в 1922 году Л.Ф. Ричардсон в своих вычислениях для предсказания погоды делил физическую область 
на ячейки и использовал конечно-разностные схемы для дискретизации уравнений \cite{richardson1922weather}.
%

Динамику потока газов и жидкости описывает система уравнений Навье-Стокса. 
%
До сих пор вопрос о существовании и единственности решения уравнений Навье-Стокса остается нерешенным. 
%
Но, несмотря на это, данные уравнения позволяют моделировать течения с хорошей точностью, 
которые согласуются с экспериментом. 
%
В работе используется метод прямого численного моделирования (DNS, от англ. Direct Numerical Simulation). 
%
Данный метод подразумевает под собой решение уравнений на всех масштабах без использования каких-либо 
дополнительных приближений и моделей. 
%
Прямое численное моделирование является концептуально самым простым методом симуляции процесса. 
%
Однако он требует больших вычислительных затрат. 
%
Поэтому для моделирования используют вычислительные кластеры и суперкомпьютеры. 
%


В данной работе рассматривается явление пристенного обратного течения в турбулентных потоках. 
%
Пристенные обратные течение – явление при котором в турбулентом потоке образуются течения, 
движущиеся против направления распространения среды.
%  
На протяжении долгого времени возникновение пристенных обратных течений считалось невозможным, 
о чем свидетельствует работа Экельмана \cite{eckelmann1974structure}. 
%
Позже их существование было показано экспериментально и в численных расчетах.
%


Пристенные обратные течения (NWRF, от англ. Near Wall Reverse Flow) могут возникать как на стенках, 
так и в углах квадратного канала. 
%
Согласно работам \cite{zaripov2021mechanism, zaripov2021reverse, ivashchenko2021effect}, 
вероятность образования NWRF в углах выше чем на стенках. 
%
Помимо того в работах \cite{zaripov2021mechanism, zaripov2021reverse} был предложен механизм 
образования пристенных обратных течений на основе данных, полученных в ходе эксперимента 
с использованием оптических методов измерения (PIV, от англ. particle image velocimetry) 
и прямого численного моделирования. 
%
Согласно данным работам NWRF образуются в результате взаимодействия набегающего потока и 
покоящейся среды через границу раздела между ними. 
%
В соответствии с широко распространенным механизмом образования пристеночной 
турбулентности \cite{robinson1991coherent}, такого рода взаимодействие приводит к 
образованию сильного сдвигового слоя на границе раздела. 
%
Из-за нестабильности сдвигового слоя формируются подковообразные вихри, 
которые создают пристенное обратное течение.
%
Планируется исследовать влияние нагрева стенок канала на процесс образование пристенных обратных течений 
и их статистические характеристики. 
%
Моделирование проводится с помощью открытого вычислительного пакета Nek5000, описанного в предыдущей главе. 
%
Предполагается, что NWRF могут стать причиной локальных перегревов в каналах, 
что может быть критично во многих инженерных приложениях.

%--------------------------------------------------------------------------------------------


Потоки в трубах и гидравлических системах, как правило, турбулентны, и потери на трение, 
возникающие в этих потоках, составляют примерно 10\% мирового потребления электроэнергии. 
%
Турбулентный режим течения сопряжен с увеличением сопротивления, и, следовательно, 
для поддержания требуемой скорости потока требуются гораздо большие энергозатраты. 
%
Несмотря на тот факт, что ламинарный режим течения считается устойчивым для всех чисел Рейнольдса, 
с увеличением скорости ламинарное состояние становится все более восприимчивым к мелким помехам. 
%
Следовательно, на практике большинство течений турбулентны при достаточно больших $Re$. 
%
В то время как устойчивость ламинарного потока была изучена очень подробно, 
мало внимания уделялось устойчивости турбулентного течения. 
%
Считается что, как только турбулентность установилась, она становится стабильной.
%


Существуют некоторые способы активного управления потоком, основной идеей которых 
является использование механизмов обратной связи для подавления выбранных компонент 
скорости или энергетических структур \cite{kuhnen2018destabilizing,kuhnen2019relaminarization}. 
%
Однако на практике такое тонкое воздействие на течение в настоящее время труднодостижимо. 
%
Именно поэтому идея о пассивном управлении турбулентным потоком выглядит наиболее перспективной, 
в частности использование особых устройств-реламинаризаторов, которые благодаря своим геометрическим 
особенностям воздействуют на определенные компоненты скорости и пульсаций скорости, 
что и приводит к дальнейшей реламинаризации всего потока.

%--------------------------------------------------------------------------------------------

%
Атомная энергия на сегодняшний день играет важную роль в производстве электроэнергии по всему миру, 
при этом, по оценкам Международного Агентства по Атомной Энергии (МАГАТЭ), 
мировые мощности атомной энергетики могут увеличиться вдвое к 2050 году \cite{gritsevskyi2016outlook}.
%
В России, по данным концерна ``Росэнергоатома'', в 2022 году доля атомной энергии составляет около 
$20\%$ от всей вырабатываемой электроэнергии в стране. 
%
При этом абсолютное большинство ядерных реакторов в мире является гетерогенными.
%
Это означает, что ядерное горючее в них конструктивно отделяется от теплоносителя и 
находится в так называемых тепловыделяющих элементах (ТВЭЛах).
%
Каждый такой элемент состоит из сердечника, в котором находится делящееся вещество, и оболочки.
% 


ТВЭЛы собираются в тепловыделяющую сборку (ТВС), представляющую собой пучок стержней, 
которая помещается в активной зоне реактора. 
%
Производительность реактора зависит от достижения равномерного распределения температуры теплоносителя внутри ТВС. 
%
Важное значение в равномерном распределении температуры теплоносителя играет турбулентная структура течения, 
формирующегося при обтекании пучков ТВЭлов, в частности в межтвэльных каналах. 
%
Исследованию течения в межтвэльном канале посвящено множество работ, из которых известно, 
что на межканальный обмен значительное влияние оказывают вихревые структуры, 
формирующиеся при обтекании пучка ТВЭЛов \cite{meyer2010discovery}. 
%
Пространственные и временные масштабы данных вихревых структур находятся 
в зависимости от плотности упаковки ТВС, определяющей ширину межтвэльного зазора.
%
Внутри ТВС выделяют две формы межтвэльного канала, относящиеся 
к внутренним и периферийным ячейкам. 
%
Внутренние ячейки образуются между рядом стоящими ТВЭЛами.
%
Периферийные образуются между рядом ТВЭЛов и плоской стенкой чехла ТВС. 
%
В связи с простотой проведения экспериментов большинство исследований структуры 
турбулентного течения проведено в периферийных ячейках.
%
Однако, различие в геометрии межтвэльных каналах внутренней и периферийной ячейки 
может сказываться на структуре турбулентного потока.
%
Таким образом для ряда задач при исследовании тепломассообмена необходимы 
обоснования использования данных, полученных в периферийных ячейках, для верификации результатов внутренних ячеек.
%
В данной работе при помощи метода прямого численного моделирования 
проводится исследование влияния формы геометрии межтвэльных каналов, 
имитирующей внутреннюю и периферийную ячейку ТВС, на структуру турбулентного потока. 

Стоит отметить, что с помощью DNS уже исследовались внутренние конфигурации 
подобных модельных сборок для различных чисел Рейнольдса \cite{shams2018towards}, 
но в данной работе упор делается именно на сравнении течения с периферийной областью.
%
%--------------------------------------------------------------------------------------------






\ifsynopsis
Этот абзац появляется только в~автореферате.
Для формирования блоков, которые будут обрабатываться только в~автореферате,
заведена проверка условия \verb!\!\verb!ifsynopsis!.
Значение условия задаётся в~основном файле документа (\verb!synopsis.tex! для
автореферата).
\else
% Этот абзац появляется только в~диссертации.
% Через проверку условия \verb!\!\verb!ifsynopsis!, задаваемого в~основном файле
% документа (\verb!dissertation.tex! для диссертации), можно сделать новую
% команду, обеспечивающую появление цитаты в~диссертации, но~не~в~автореферате.
\fi

% {\progress}
% Этот раздел должен быть отдельным структурным элементом по
% ГОСТ, но он, как правило, включается в описание актуальности
% темы. Нужен он отдельным структурынм элемементом или нет ---
% смотрите другие диссертации вашего совета, скорее всего не нужен.

{\aim} данной работы является исследование различных пристеночных течений 
методом прямого численного моделирования.

Для~достижения поставленной цели необходимо было решить следующие {\tasks}:
\begin{enumerate}[beginpenalty=10000] % https://tex.stackexchange.com/a/476052/104425
  \item Исследовать обратные пристеночные течения в квадратном канале при нагреве стенок 
  \item Исследовать пассивные методы реламинаризации турбулентного течения в круглой трубе
  \item Исследовать две конфигурации модельной сборки тепловыделяющих элементов
\end{enumerate}


{\novelty}
\begin{enumerate}[beginpenalty=10000] % https://tex.stackexchange.com/a/476052/104425
  \item Впервые получены результаты по влиянию нагрева стенок квадратного канала на вероятность 
  возникновения обратных пристеночных течений
  \item Предложен метод моделирования устройств-реламинаризаторов в трубопроводах
  \item Показано отличие течений в различных участках тепловыделяющей сборки
\end{enumerate}

{\influence} Работа вносит вклад в развитие методов численного моделирования сдвиговых течений. 
Результаты, полученные автором, имеют как фундаментальную значимость, так и практическую. 

Полученные знания о механизме возникновения обратных пристеночных течений имеют большую значимость в 
задачах теплоотвода от плоской стенки. Результаты, полученные по реламинаризации турбулентного течения в круглых
трубах, можно применить для совершенствования трубопроводного способа транспортировки природных ресурсов.
Данные по различиям течений во внутренней и периферийной областях тепловыделяющей сборки используются для 
верификации экспериментальных данных, а также для создания новых RANS моделей турбулентности.

%{\methods} \ldots

{\defpositions}
\begin{enumerate}[beginpenalty=10000] % https://tex.stackexchange.com/a/476052/104425
  \item Результаты исследования обратных пристеночных течений в квадратном канале при наличии нагрева стенок
  \item Результаты исследования пассивных способов реламинаризации турбулентного течения в круглой трубе
  \item Результаты исследовния обтекания внутренней и периферийной конфигураций модельной сборки 
  тепловыделяющих элементов
\end{enumerate}

{\reliability} полученных результатов обеспечивается
использованием проверенных вычислительных алгоритмов в открытом пакете Nek5000,
верификацией полученных результатов с помощью экспериментальных данных, а также
воспроизводимостью полученных данных.

{\contribution} Основные научные результаты, включенные в диссертацию и 
выносимые автором на защиту получены соискателем самостоятельно. Постановка задач исследования
и научная проблематика разрабатывалась диссертантом как единолично, так и при участии
д.ф.-м.н. Мулляджанова Р.И., что обеспечило комплексный подход к изучению темы.
В опубликованных совместных работах личный вклад автора заключается в проведении
численных экспериментов методом прямого численного моделирование, модернизации блоков 
расчетного кода, обработке полученных данных, а также подготовке научных докладов и 
публикаций в рецензируемых журналах.

Соискатель является исполнителем исселодовательских проектов по тематике диссертационной работы.
Представление изложенных в диссретации и выносимых на защиту результатов, полученных в 
совместных исследованиях, согласовано с соавторами.


{\probation}
Основные результаты работы докладывались~на ведущих российских и международных конференциях:
Международной научной студенческой конференции (Новосибирск, 2016, 2018, 2019),
Всероссийской научной конференции ``Теплофизика и физическая гидродинамика'' (Ялта, 2016, 2018; Сочи, 2022),
Всероссийской школе-конференции молодых ученых с международным участием ``Актуальные вопросы теплофизики и физической гидрогазодинамики'' (Новосибирск, 2016; Шерегеш 2023),
Международной конференции по методам аэрофизических исследований ``ICMAR''(Новосибирск, 2018, 2020),
Всероссийской конференции с элементами научной школы для молодых учёных ``Сибирский теплофизический семинар'' (Новосибирск, 2018--2022),
17th European Turbulence Conference, (Турин, 2019),
XV Всероссийской школе-конференции молодых ученых ``Проблемы механики: теория, эксперимент и новые технологии'' (Шерегеш, 2021),
Международной конференции ``Марчуковские научные чтения'' (Новосибирск, 2021),
14th International Symposium on Particle Image Velocimetry, (США, 2021),
Всероссийской школе-конференции НЦФМ ``Математическое моделирование на супер-ЭВМ экса- и зеттафлопсной производительности'' (Саров, 2022).



% \ifnumequal{\value{bibliosel}}{0}
% {%%% Встроенная реализация с загрузкой файла через движок bibtex8. (При желании, внутри можно использовать обычные ссылки, наподобие `\cite{vakbib1,vakbib2}`).
{\publications} Основные результаты по теме диссертации изложены
    в~37~печатных изданиях,
    15 из которых изданы в журналах, рекомендованных ВАК,
    22 "--- в тезисах докладов.
% }%
% {%%% Реализация пакетом biblatex через движок biber
%     \begin{refsection}[bl-author, bl-registered]
%         % Это refsection=1.
%         % Процитированные здесь работы:
%         %  * подсчитываются, для автоматического составления фразы "Основные результаты ..."
%         %  * попадают в авторскую библиографию, при usefootcite==0 и стиле `\insertbiblioauthor` или `\insertbiblioauthorgrouped`
%         %  * нумеруются там в зависимости от порядка команд `\printbibliography` в этом разделе.
%         %  * при использовании `\insertbiblioauthorgrouped`, порядок команд `\printbibliography` в нём должен быть тем же (см. biblio/biblatex.tex)
%         %
%         % Невидимый библиографический список для подсчёта количества публикаций:
%         \printbibliography[heading=nobibheading, section=1, env=countauthorvak,          keyword=biblioauthorvak]%
%         \printbibliography[heading=nobibheading, section=1, env=countauthorwos,          keyword=biblioauthorwos]%
%         \printbibliography[heading=nobibheading, section=1, env=countauthorscopus,       keyword=biblioauthorscopus]%
%         \printbibliography[heading=nobibheading, section=1, env=countauthorconf,         keyword=biblioauthorconf]%
%         \printbibliography[heading=nobibheading, section=1, env=countauthorother,        keyword=biblioauthorother]%
%         \printbibliography[heading=nobibheading, section=1, env=countregistered,         keyword=biblioregistered]%
%         \printbibliography[heading=nobibheading, section=1, env=countauthorpatent,       keyword=biblioauthorpatent]%
%         \printbibliography[heading=nobibheading, section=1, env=countauthorprogram,      keyword=biblioauthorprogram]%
%         \printbibliography[heading=nobibheading, section=1, env=countauthor,             keyword=biblioauthor]%
%         \printbibliography[heading=nobibheading, section=1, env=countauthorvakscopuswos, filter=vakscopuswos]%
%         \printbibliography[heading=nobibheading, section=1, env=countauthorscopuswos,    filter=scopuswos]%
%         %
%         \nocite{*}%
%         %
%         {\publications} Основные результаты по теме диссертации изложены в~\arabic{citeauthor}~печатных изданиях,
%         \arabic{citeauthorvak} из которых изданы в журналах, рекомендованных ВАК\sloppy%
%         \ifnum \value{citeauthorscopuswos}>0%
%             , \arabic{citeauthorscopuswos} "--- в~периодических научных журналах, индексируемых Web of~Science и Scopus\sloppy%
%         \fi%
%         \ifnum \value{citeauthorconf}>0%
%             , \arabic{citeauthorconf} "--- в~тезисах докладов.
%         \else%
%             .
%         \fi%
%         \ifnum \value{citeregistered}=1%
%             \ifnum \value{citeauthorpatent}=1%
%                 Зарегистрирован \arabic{citeauthorpatent} патент.
%             \fi%
%             \ifnum \value{citeauthorprogram}=1%
%                 Зарегистрирована \arabic{citeauthorprogram} программа для ЭВМ.
%             \fi%
%         \fi%
%         \ifnum \value{citeregistered}>1%
%             Зарегистрированы\ %
%             \ifnum \value{citeauthorpatent}>0%
%             \formbytotal{citeauthorpatent}{патент}{}{а}{}\sloppy%
%             \ifnum \value{citeauthorprogram}=0 . \else \ и~\fi%
%             \fi%
%             \ifnum \value{citeauthorprogram}>0%
%             \formbytotal{citeauthorprogram}{программ}{а}{ы}{} для ЭВМ.
%             \fi%
%         \fi%
%         % К публикациям, в которых излагаются основные научные результаты диссертации на соискание учёной
%         % степени, в рецензируемых изданиях приравниваются патенты на изобретения, патенты (свидетельства) на
%         % полезную модель, патенты на промышленный образец, патенты на селекционные достижения, свидетельства
%         % на программу для электронных вычислительных машин, базу данных, топологию интегральных микросхем,
%         % зарегистрированные в установленном порядке.(в ред. Постановления Правительства РФ от 21.04.2016 N 335)
%     \end{refsection}%
%     \begin{refsection}[bl-author, bl-registered]
%         % Это refsection=2.
%         % Процитированные здесь работы:
%         %  * попадают в авторскую библиографию, при usefootcite==0 и стиле `\insertbiblioauthorimportant`.
%         %  * ни на что не влияют в противном случае
%         \nocite{vakbib2}%vak
%         \nocite{patbib1}%patent
%         \nocite{progbib1}%program
%         \nocite{bib1}%other
%         \nocite{confbib1}%conf
%     \end{refsection}%
%         %
%         % Всё, что вне этих двух refsection, это refsection=0,
%         %  * для диссертации - это нормальные ссылки, попадающие в обычную библиографию
%         %  * для автореферата:
%         %     * при usefootcite==0, ссылка корректно сработает только для источника из `external.bib`. Для своих работ --- напечатает "[0]" (и даже Warning не вылезет).
%         %     * при usefootcite==1, ссылка сработает нормально. В авторской библиографии будут только процитированные в refsection=0 работы.
% }

% При использовании пакета \verb!biblatex! будут подсчитаны все работы, добавленные
% в файл \verb!biblio/author.bib!. Для правильного подсчёта работ в~различных
% системах цитирования требуется использовать поля:
% \begin{itemize}
%         \item \texttt{authorvak} если публикация индексирована ВАК,
%         \item \texttt{authorscopus} если публикация индексирована Scopus,
%         \item \texttt{authorwos} если публикация индексирована Web of Science,
%         \item \texttt{authorconf} для докладов конференций,
%         \item \texttt{authorpatent} для патентов,
%         \item \texttt{authorprogram} для зарегистрированных программ для ЭВМ,
%         \item \texttt{authorother} для других публикаций.
% \end{itemize}
% Для подсчёта используются счётчики:
% \begin{itemize}
%         \item \texttt{citeauthorvak} для работ, индексируемых ВАК,
%         \item \texttt{citeauthorscopus} для работ, индексируемых Scopus,
%         \item \texttt{citeauthorwos} для работ, индексируемых Web of Science,
%         \item \texttt{citeauthorvakscopuswos} для работ, индексируемых одной из трёх баз,
%         \item \texttt{citeauthorscopuswos} для работ, индексируемых Scopus или Web of~Science,
%         \item \texttt{citeauthorconf} для докладов на конференциях,
%         \item \texttt{citeauthorother} для остальных работ,
%         \item \texttt{citeauthorpatent} для патентов,
%         \item \texttt{citeauthorprogram} для зарегистрированных программ для ЭВМ,
%         \item \texttt{citeauthor} для суммарного количества работ.
% \end{itemize}
% % Счётчик \texttt{citeexternal} используется для подсчёта процитированных публикаций;
% % \texttt{citeregistered} "--- для подсчёта суммарного количества патентов и программ для ЭВМ.

% Для добавления в список публикаций автора работ, которые не были процитированы в
% автореферате, требуется их~перечислить с использованием команды \verb!\nocite! в
% \verb!Synopsis/content.tex!.
